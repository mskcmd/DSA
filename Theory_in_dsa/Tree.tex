# Tree

Tree:  a tree is a hierarchical data structure that consists of nodes connected by edges

**Types of trees in DSA**

1. **Binary Tree:** A tree in which each node has at most two children, referred to as the left child and the right child.
2. **AVL Tree:** A self-balancing binary search tree in which the heights of the two child subtrees of any node differ by at most one.
3. **B-Tree:** A tree data structure that is designed to work well with block storage devices and databases. It is balanced and allows for efficient searching, insertion, and deletion.
4. **Trie (Prefix Tree):** A tree-like data structure used to store a dynamic set or associative array where the keys are usually strings.

AVL Tree ➖

**Traversal** 

Tree types of traversals:

1. **In-order Traversal:** Visit the left subtree, then the root, and finally the right subtree. For a BST, this results in visiting nodes in ascending order.
2. **Pre-order Traversal:** Visit the root first, followed by the left subtree, and then the right subtree.
3. **Post-order Traversal:** Visit the left subtree, then the right subtree, and finally the root.
4. **Level-order Traversal:** Visit nodes level by level, starting from the root and moving to the next level.

 **Common types of binary trees:**

1. **Full Binary Tree (Proper Binary Tree):**
    - Every node in the tree has either 0 or 2 children.
    - No node has only one child.
2. **Complete Binary Tree:**
    - All levels of the tree are completely filled, except possibly for the last level.
    - The last level is filled from left to right.
3. **Perfect Binary Tree:**
    - All levels of the tree are completely filled.
    - The number of nodes is \(2^h - 1\), where \(h\) is the height of the tree.
4. **Balanced Binary Tree:**
    - The height of the left and right subtrees of any node differs by at most one.
    - Ensures that the tree remains balanced, facilitating efficient search operations.
5. **Degenerate (or pathological) tree:**
    - A tree where each parent node has only one associated child node.
    - Essentially, it is a linked list.
6. **Skewed Binary Tree:**
    - A special case of a degenerate tree is one where all nodes have only one child and the tree is either left-skewed or right-skewed.
7. **Expression Tree:**
    - A binary tree is used to represent expressions, where the leaves are operands and the internal nodes are operators.
8. **Threaded Binary Tree:**
    - A binary tree is one in which every node has either one or two threads (pointers to other nodes), making traversal more efficient.

**BFS (Breadth-First Search) vs. DFS (Depth-First Search):**

**BFS (Breadth-First Search):**

- **Traversal Order:** Visit nodes level by level, exploring all the neighbors of a node before moving on to the next level.
- **Data Structure:** Typically implemented using a queue.
- **Use Cases:**
    - Shortest path problems.
    - Finding connected components in an undirected graph.
    - Solving puzzles like mazes.

**DFS (Depth-First Search):**

- **Traversal Order:** Explore as far as possible along each branch before backtracking.
- **Data Structure:** Typically implemented using recursion or a stack.
- **Use Cases:**
    - Topological sorting.
    - Finding connected components in an undirected graph.
    - Solving puzzles like mazes.

**Types of BFS:**

1. **Standard BFS:** Visit nodes in the order they are discovered, level by level.
2. **Bidirectional BFS:** Start BFS from both the source and destination simultaneously, meeting in the middle.

**Types of DFS:**

1. **Pre-order DFS:** Visit the root, then recursively do a pre-order DFS on the left and right subtrees.
2. **In-order DFS:** Recursively do an in-order DFS on the left subtree, visit the root, and then do an in-order DFS on the right subtree.
3. **Post-order DFS:** Recursively do a post-order DFS on the left and right subtrees, then visit the root.

**Time Complexity:**

- The time complexity of both BFS and DFS is \(O(V + E)\), where \(V\) is the number of vertices (nodes) and \(E\) is the number of edges in the graph.
- In the worst case, both algorithms may need to visit all vertices and edges.

**Space Complexity:**

- The space complexity of BFS is generally higher due to the use of a queue, making it \(O(V)\) in the worst case.
- The space complexity of DFS is often lower, typically \(O(\log V)\) for a balanced tree or \(O(V)\) for a skewed tree.

**What is a Tree?**
A tree is a hierarchical data structure that consists of nodes connected by edges, where each node has a parent and zero or more children. The topmost node is called the root, and nodes with no children are called leaves. Trees are widely used in computer science for organizing and representing hierarchical relationships between entities.

**Types of Trees:**

1. **Binary Tree:** Each node has at most two children.
2. **Binary Search Tree (BST):** A binary tree with the property that the value of each node's left child is less than or equal to the node's value, and the value of the right child is greater than or equal to the node's value.
3. **AVL Tree:** A self-balancing binary search tree that maintains a balance factor to ensure logarithmic time complexity for search, insertion, and deletion.
4. **B-Tree:** A self-balancing search tree designed for use in databases and file systems.
5. **Trie (Prefix Tree):** A tree structure used to store a dynamic set or associative array where keys are usually strings.
6. **Red-Black Tree:** A self-balancing binary search tree that ensures balanced height, maintaining logarithmic time complexity for search, insertion, and deletion.
7. **Quadtree:** A tree data structure used in computer graphics and geographic information systems to represent two-dimensional space.

**Tree Operation Time Complexity:**

1. **Search (in a balanced tree):** O(log n), where n is the number of nodes.
2. **Insertion (in a balanced tree):** O(log n).
3. **Deletion (in a balanced tree):** O(log n).
4. **Traversal (e.g., in-order, pre-order, post-order):** O(n), where n is the number of nodes.

**Merits of Trees:**

1. **Hierarchical Organization:** Trees naturally represent hierarchical relationships.
2. **Efficient Search Operations:** Balanced trees provide efficient search, insertion, and deletion operations.
3. **Versatility:** Different types of trees suit various applications and requirements.
4. **Ease of Representation:** Trees can represent hierarchical structures like file systems, organizational charts, etc.

**Demerits of Trees:**

1. **Complexity:** Some types of trees, such as AVL trees or B-trees, may have complex implementation and maintenance.
2. **Memory Usage:** Trees can consume more memory compared to flat data structures.
3. **Not Always Optimal:** Trees might not be the optimal choice for certain applications, depending on the specific requirements.

**Tree Real-Time Applications:**

1. **File Systems:** Trees represent the hierarchical structure of file systems.
2. **Database Indexing:** Trees, such as B-trees, are used for efficient database indexing.
3. **Organizational Charts:** Trees can represent hierarchical organizational structures.
4. **Routing Tables:** Tries are used in networking for efficient IP routing.
5. **Computer Graphics:** Quadtrees are used in computer graphics for spatial partitioning.

**Tree vs. Array:**

1. **Structure:** Trees are hierarchical structures, while arrays are linear structures.
2. **Search:** Trees provide efficient search operations (O(log n)), while arrays have constant-time access (O(1)) but may require linear search in unsorted arrays.
3. **Insertion/Deletion:** Trees have efficient insertion and deletion (O(log n)), while arrays may require shifting elements for insertions and deletions.
4. **Dynamic Size:** Trees can dynamically grow or shrink, while array size is typically fixed.
5. **Applications:** Trees are used for hierarchical relationships, whereas arrays are often used for sequential data storage.

The choice between trees and arrays depends on the specific requirements of the application, the nature of the data, and the desired operations' efficiency.
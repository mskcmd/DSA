# DSA 2

1.**Data Structures:** These are specialized formats for organizing and storing data to perform operations efficiently. Examples include arrays, linked lists, stacks, queues, trees, and hash tables.

**2. Array:** An array is a data structure that stores elements of the same type in contiguous memory locations. The elements are accessed using an index or a key.

**3. Types of DSA:**

- **Data Structures:** Arrays, Linked Lists, Trees, Graphs, Hash Tables, etc.
- **Algorithms:** Sorting, Searching, Recursion, Dynamic Programming, etc.

**4. Algorithm:** An algorithm is a step-by-step procedure or formula for solving a specific problem. It is a set of rules or instructions that describe how to perform a particular task.

**5. Big-O Notation:** Big-O notation is used to describe the upper bound of an algorithm's time complexity or space complexity in terms of the input size. It provides an asymptotic upper bound to express the efficiency of an algorithm.

**6. Complexity Analysis:** Complexity analysis involves evaluating the efficiency of algorithms in terms of time and space complexity. It helps in comparing different algorithms and choosing the most suitable one for a specific problem.

**7. Time Complexity:** Time complexity measures the amount of time an algorithm takes to complete concerning the input size.

**8. Space Complexity:** Space complexity measures the amount of memory an algorithm uses concerning the input size.

**9. Queue (with example):** A queue is a data structure that follows the First-In-First-Out (FIFO) principle. Elements are inserted at the rear (enqueue) and removed from the front (dequeue). Example: A printer queue where print jobs are processed in the order they are received.

**10. Stack (with example):** A stack is a data structure that follows the Last-In-First-Out (LIFO) principle. Elements are added and removed from the same end. Example: The call stack in programming where function calls are managed.

**11. Memory Leak:** A memory leak occurs when a program allocates memory but fails to release it properly, leading to a gradual depletion of available memory.

**12. Garbage Collection:** Garbage collection is a process in programming languages where the runtime system automatically reclaims memory occupied by objects that are no longer in use.

**13. Memory Allocation:** Memory allocation is the process of setting aside blocks of memory for a program's use.

**14. Linked List:** A linked list is a linear data structure where elements are stored in nodes, and each node points to the next one in the sequence.

**15. Head and Tail (in linked list):** The head is the first node in a linked list, and the tail is the last node. The tail's next pointer typically points to null.

**16. Doubly Linked List:** In a doubly linked list, each node contains a data element and two pointers, one pointing to the next node and another pointing to the previous node. 

**17. Singly Linked List:** In a singly linked list, each node contains a data element and a pointer to the next node in the sequence.

**18. Concepts of String:** Strings are sequences of characters. Concepts include manipulation, concatenation, comparison, and various string-related operations.

**19. Linear Search & Binary Search (in JavaScript):**

- **Linear Search:** Iterates through each element until the target is found.
- **Binary Search:** Assumes a sorted array and repeatedly divides the search range in half.

Feel free to ask for more clarification on any specific topic!